\documentclass[a4,oneside]{article}
\usepackage[colorlinks=true,linkcolor=blue,citecolor=blue]{hyperref}
\usepackage{natbib}
\usepackage{times}
\usepackage{listings}
\usepackage[table]{xcolor}
%\PassOptionsToPackage{table}{xcolor}
\usepackage{color}
%\usepackage{colortbl}
\usepackage{marginnote}
\usepackage{rotating}
\usepackage{lipsum}
\usepackage{longtable}
\usepackage{array}
\usepackage{amsmath}
\usepackage{mdframed,lipsum}
\newcolumntype{L}{>{\raggedright\arraybackslash\hangindent=1em}}
\newcolumntype{M}{>{\raggedright\arraybackslash\hangindent=1em\ttfamily}}
\newcolumntype{X}{>{\nullfont}c}
\def\tablesetup{\rowcolors{2}{light-gray}{light-gray}\footnotesize}
\newmdenv[
  leftmargin = 0pt,
  innerleftmargin = 1em,
  innertopmargin = 0pt,
  innerbottommargin = 0pt,
  innerrightmargin = 0pt,
  rightmargin = 0pt,
  linewidth = 1pt,
  topline = false,
  rightline = false,
  bottomline = false
  ]{leftbar}
%\DeclareTextCommand{\_}{OT1}{\leavevmode\vbox{\hrule width.5em}}
% Use proper underscore character
\chardef\_=`\_
% Set math in Times Roman
\DeclareSymbolFont{letters}{OML}{ptmcm}{m}{it}
\DeclareSymbolFont{operators}{OT1}{ptmcm}{m}{n}
%\DeclareSymbolFont{bold}     {OML}{ptmcm}{b}{it}
\DeclareMathAlphabet{\mathbf}{OT1}{ptm}{b}{n}
% Page set up
\setlength{\oddsidemargin}{0cm} %{0.5cm}
\setlength{\evensidemargin}{0cm} %{0.5cm}
\setlength{\topmargin}{-2cm}
\setlength{\textheight}{24cm}
\setlength{\textwidth}{16cm}
\setlength{\marginparsep}{0.5cm}
\setlength{\marginparwidth}{0cm}
\setlength{\parindent}{1em}
\setlength{\parskip}{0cm}
\renewcommand{\baselinestretch}{1.1}
\sloppy

% Configure appearance of code listings
\definecolor{light-gray}{gray}{0.92}
\def\codesize{\small}
\def\codetabsize{\footnotesize}
\lstset{language=Fortran,
  backgroundcolor=\color{light-gray},
  basicstyle=\footnotesize\ttfamily,
  numbersep=5pt,
  xleftmargin=0cm,
  xrightmargin=0cm,
  emph={true,false,include,to},
  emphstyle=\relax}
\lstset{showstringspaces=false}

% Table-of-contents configuration
\usepackage{tocloft}
\setlength\cftparskip{-2pt}
\setlength\cftbeforesecskip{1pt}
\setlength\cftaftertoctitleskip{2pt}
\renewcommand\cftsecfont{\normalfont}
\renewcommand\cftsecpagefont{\normalfont}
\renewcommand{\cftsecleader}{\cftdotfill{\cftsecdotsep}}
\renewcommand\cftsecdotsep{\cftdot}
\renewcommand\cftsubsecdotsep{\cftdot}

% Page headers
\usepackage{fancyhdr}
\pagestyle{fancy}
\renewcommand{\headrulewidth}{0.5pt}
\renewcommand{\sectionmark}[1]{\markright{\thesection.\ #1}}
\renewcommand{\subsectionmark}[1]{}
\fancyhead[RO,RE]{\thepage}
\fancyfoot[C]{}

% Symbols and macros
\def\spsurf{\emph{SPARTACUS-Surface}}
\def\code#1{{\codesize\texttt{#1}}}
\def\codetab#1{{\codetabsize\texttt{#1}}}
\def\codeemph#1{{\codesize\texttt{\textbf{#1}}}}
\def\codetabemph#1{{\codetabsize\texttt{\textbf{#1}}}}
\def\textemph#1{\textbf{#1}}
\def\citem#1{\item[{\codesize\texttt{#1}}]}
\def\codestyle#1{\texttt{#1}}
\renewcommand\thefootnote{\relax}
\def\chapter{\section}
\reversemarginpar

% Title material
\title{SPARTACUS-Surface User Guide}

\author{Robin J. Hogan\\ \emph{European Centre for Medium Range
    Weather Forecasts, Reading, UK}}

\date{Document version 0.6 (February 2020) applicable to
  \spsurf\ version 0.6\thanks{This document is copyright
    \copyright\ European Centre for Medium Range Weather Forecasts
    2019. If you have any queries about \spsurf\ that are not
    answered by this document or by the information on the \spsurf\ web
    site
    (\url{http://www.met.reading.ac.uk/clouds/spartacus})
    then please email me at
    \href{mailto:r.j.hogan@ecmwf.int}{\texttt{r.j.hogan@ecmwf.int}}.}}
\begin{document}
\maketitle

%\tableofcontents
\def\thefootnote{\fnsymbol{footnote}}
\chapter{Introduction}
%\section{What is \spsurf?}
\spsurf\ is a Fortran-2003 software library for computing the 3D
interaction of solar (or \emph{shortwave}) and thermal-infrared (or
\emph{longwave}) radiation with complex surface canopies, especially
forests and urban areas. It uses a multi-layer description of the
canopy but with a statistical description of the horizontal
distribution of trees and buildings. This greatly reduces the
variables needed to describe the canopy, and makes the scheme fast
enough to use in weather and climate models.

The detailed theoretical basis of the library is provided in two
papers: \cite{Hogan+2018} developed the shortwave forest solver, and
\cite{Hogan2019} extended this to include buildings and longwave
radiative transfer.\footnote{These works developed two prototype codes
  in Matlab: \emph{SPARTACUS-Vegetation} and \emph{SPARTACUS-Urban},
  both available from the SPARTACUS web site.}  The resulting
algorithm combines three key ideas from earlier papers in the
atmospheric radiative transfer literature:
%
\begin{itemize}
\item To represent horizontal variations in vegetation leaf density
  (or equivalently, extinction coefficient), each layer in a
  vegetation canopy is divided horizontally into three regions:
  clear-air (unvegetated) and two vegetated regions of equal
  fractional cover but different extinction coefficient.  This
  approach was proposed by \cite{Shonk+2008}, who showed (in the
  context of cloudy radiative transfer) that the radiative effect of
  the full distribution of extinction coefficient could be
  approximated well given an appropriate choice for the extinction
  coefficients of the two regions.
\item Three-dimensional radiative effects are treated rigorously by
  using the \emph{Speedy Algorithm for Radiative Transfer through
    Cloud Sides} (SPARTACUS) of \cite{Hogan+2016}, but replacing
  clouds with trees and buildings.  Since it is reasonable to treat
  trees and buildings as randomly distributed in the horizontal plane,
  the rate of exchange of radiation between the clear and vegetated
  parts of a layer may be assumed to be proportional to the length of
  the interface between them, and likewise for the rate of
  interception of radiation by building walls.
\item The \emph{Discrete Ordinate Method} is used to approximate the
  zenithal distribution of diffuse radiation, with the coupled
  ordinary differential equations solved by Eigen decomposition
  similarly to \cite{Stamnes+1989}. This is more robust than the
  matrix-exponential method used by \cite{Hogan+2016}, and more
  accurate since the diffuse radiation field may be described by more
  than just two streams.
\end{itemize}

Chapter \ref{ch:offline} describes how to compile and use the offline
version of \spsurf, which is essentially a Unix program that reads a
configuration file and a netCDF file containing a description of a
number of surfaces, and outputs a netCDF file containing the computed
radiation properties. Chapter \ref{ch:api} describes how to incorporate
\spsurf\ into a larger Fortran program, such as an atmospheric model.

%Chapter \ref{ch:api} describes the Application Programming Interface
%(API) enabling \spsurf\ to be incorporated into a larger Fortran
%program such as an atmospheric model. Chapter \ref{ch:structure}
%describes the internal architecture of the \spsurf\ software. Chapter
%\ref{ch:science} provides the scientific documentation, including the
%equations solved by \spsurf.

%\chapter{Using the offline radiation scheme}
%\label{ch:offline}
\section{Compiling the package}
The offline version of \spsurf\ is designed to be used on a Unix-like
platform. You will need a Fortran compiler that supports the 2003
standard, such as \code{gfortran}.
%
As a prerequisite, you will need to install the netCDF library,
including the Fortran interface (packages to install on a Linux system
are typically called \code{libnetcdff-dev} or
\code{libnetcdff-devel}).  If you have a recent netCDF version then
the command
\begin{lstlisting}
 nc-config --fc
\end{lstlisting}
should return the Fortran compiler for which the netCDF library was
compiled.  To run some of the tests, you will also need to install the
NCO utilities for manipulating netCDF Files.

First unpack the package and enter the subdirectory as follows:
\begin{lstlisting}
 tar xvfz spartacus_surface-0.8.tar.gz
 cd spartacus_surface-0.8
\end{lstlisting}
The latest snapshot may also be obtained from GitHub.  On a non-GNU
platform you may need to untar and unzip the package using the
\code{tar} and \code{gunzip} commands separately. The \code{README}
file contains concise instructions on compilation and testing, while
the \code{COPYING} file provides the license conditions (Apache
License, version 2.0). The subdirectories are as follows:
%
\begin{description}
\citem{radsurf} The \spsurf\ souce code for canopy radiative transfer
\citem{radtool} Mathematical support routines for radiative transfer
\citem{ifsaux} Source code providing a (sometimes dummy) IFS environment
\citem{utilities} Source code for useful utilities, such as reading netCDF
       files
\citem{driver} The source code for the offline driver program \code{spartacus\_surface}
\citem{mod} Where Fortran module files are written
\citem{lib} Where the static libraries are written
\citem{bin} Where the executable \code{ecrad} is written
\citem{test} Test cases including Matlab code to plot the outputs
\end{description}

Compilation on different platforms using different compilers is
facilitated by the various \code{Makefile\_include.<prof>} files in the
top-level directory: if you type
%
\begin{lstlisting}
 make
\end{lstlisting}
%
or
%
\begin{lstlisting}
 make PROFILE=gfortran
\end{lstlisting}
%
the code will be compiled using the \code{gfortran} compiler via the
Makefile variables set in the \code{Makefile\_include.gfortran}
file. Using instead \code{PROFILE=pgi} will use the
\code{Makefile\_include.pgi} file to attempt to compile with the PGI
compiler, while \code{PROFILE=intel} selects the Intel compiler.  If
everything goes to plan this should create the executable
\code{bin/spartacus\_surface} and various static libraries in the \code{lib}
directory.

One common reason the code doesn't compile out of the box is that it
can't find the netCDF library files.  The \spsurf\ Makefile uses the
\code{nf-config} script that comes with recent versions of the netCDF
library to create the Makefile variables \code{NETCDF\_INCLUDE} and
\code{NETCDF\_LIB}. If \code{nf-config} is not available on your
system, or it fails to correctly locate the netCDF library files, then
the cleanest way to fix this is to create a
\code{Makefile\_include.local} file (starting from one of the existing
\code{Makefile\_include.*} files) that defines \code{NETCDF\_INCLUDE}
and \code{NETCDF\_LIB} explicity to contain arguments for the compile
and link operations, respectively.  Suppose you installed netCDF in
\code{/path/to/netcdf} and you use the \code{gfortran} compiler then
your file might contain:
\begin{lstlisting}
 include Makefile_include.gfortran
 NETCDF = /path/to/netcdf
 NETCDF_INCLUDE = -I$(NETCDF)/include
 NETCDF_LIB = -L$(NETCDF)/lib -lnetcdff -lnetcdf -Wl,-rpath,$(NETCDF)/lib
\end{lstlisting}
You should then be able to build the code with
%
\begin{lstlisting}
 make PROFILE=local
\end{lstlisting}
%
Examples of such configurations for the ECMWF and University of
Reading computer systems may be found in
\code{Makefile\_include.ecmwf} and \code{Makefile\_include.uor}.

%To compile in single precision, type
%\begin{lstlisting}
% make PROFILE=gfortran SINGLE_PRECISION=1
%\end{lstlisting}
To compile with debugging options turned on (no optimization, bounds
checking and initializing real numbers with not-a-number), type
\begin{lstlisting}
 make PROFILE=gfortran DEBUG=1
\end{lstlisting}
Finer tuning may be achieved by overriding the optimization and
debugging flags used in \code{Makefile} explicitly, for example
\begin{lstlisting}
 make PROFILE=gfortran OPTFLAGS="-O1" DEBUGFLAGS="-g1 -pg"
\end{lstlisting}
Remember that if you change the compile settings you will probably
want to recompile everything, in which case you first need to remove
all compiled files with
\begin{lstlisting}
 make clean
\end{lstlisting}

\section{Running the offline scheme}
 To test the code, type
\begin{lstlisting}
 make test
\end{lstlisting}
which runs \code{make} in each of the subdirectories of the
\code{test} directory. The \code{README} files in these directories
provide more information on what they are doing, and some Matlab
scripts are provided to visualize the outputs.

You will see in the output of the tests the command line in each
invocation of \spsurf, which is of the form
%
\begin{lstlisting}
 spartacus\_surface config.nam input.nc output.nc
\end{lstlisting}
where \code{spartacus\_surface} needs to be the full path to the
\spsurf\ executable, \code{config.nam} is a Fortran namelist file
configuring the code, \code{input.nc} contains the input atmospheric
profiles and \code{output.nc} contains the output irradiance (flux)
profiles.  The namelist file contains a \code{radsurf} namelist that
configures the \spsurf\ scheme itself; the parameters available are
described in section \ref{sec:nam_radsurf}. The file also contains a
\code{radsurf\_config} namelist that configures aspects of the offline
package, described in section \ref{sec:nam_radsurf_config}.  Only the
\code{radsurf} namelist is used when \spsurf\ is incorporated into an
atmospheric model.

The input netCDF file contains numerous floating-point variables
listed in Table \ref{tab:invar}. The dimensions are shown in the order
that they are listed by the \code{ncdump} utility, with the first
dimension varying slowest in the file (opposite to the Fortran
convention).  Most variables are stored as a function of column and
layer (dimensions named \code{col} and \code{layer} in Table
\ref{tab:invar}, although the actual dimension names are ignored by
\spsurf). The \code{layer\_int} dimension corresponds to interfaces
between layers, plus the top-of-canopy and surface, and so must be one
more than \code{layer}. Note that both \code{layer} and
\code{layer\_int} should increase upwards from the surface. The
optional \code{sw} and \code{lw} dimensions allow for shortwave and
longwave optical properties of leaves and facets to be specified in
user-defined spectral intervals. Some variables can be omitted in
which case default values will be used or these fields will be
constructed according to \code{radsurf\_config} namelist parameters
(section \ref{sec:nam_radsurf_config}).

\begin{center}
\tablesetup
\begin{longtable}{llLp{6.6cm}}%
\caption{\label{tab:invar}Main variables contained in the input netCDF
  file to \spsurf. Note that some variables are not required if they
  are not used by the particular solver selected, for example
  \code{iseed} is only used by the McICA solver and
  \code{inv\_cloud\_effective\_size} is only used by the SPARTACUS
  solver. Also, only one of \code{o3\_mmr} and \code{o3\_vmr} should
  be provided. In addition to ozone, further gases can be specified in
  either mass mixing ratio (suffix \code{\_mmr}) or volume mixing
  ratio (suffix \code{\_vmr}) units, where the prefixes are \code{co2}
  (carbon dioxide), \code{n2o} (nitrous oxide), \code{co} (carbon
  monoxide), \code{ch4} (methane), \code{o2} (molecular oxygen),
  \code{cfc11} (CFC-11), \code{cfc12} (CFC-12), \code{hcfc22}
  (HCFC-22), \code{ccl4} (carbon tetrachloride) and \code{no2}
  (nitrogen dioxide). These further trace gases may either be
  specified as variable in space (dimensioned \code{col,level}) or
  constant (a scalar value in the file). To override the suffix
  indicating volume mixing ratio (e.g.\ to change it to
  \code{\_mole\_fraction}), set the namelist variable
  \code{vmr\_suffix\_str} as described in Table
  \ref{tab:nam_radiation_config}.}\\
%
\hline
Variable & Dimensions & Description \\
\hline
\codetab{solar\_irradiance} & \codetab{-} & Solar irradiance at Earth's orbit (W~m$^{-2}$) \\
\codetab{skin\_temperature} & \codetab{col} & Skin temperature (K) \\
\codetab{cos\_solar\_zenith\_angle} & \codetab{col} & Cosine of solar zenith angle \\
\codetab{sw\_albedo} & \codetab{col, sw\_albedo\_band} & Shortwave albedo (if 1D then assumed spectrally constant) \\
\codetab{lw\_emissivity} & \codetab{col, lw\_emiss\_band} & Longwave emissivity (if 1D then assumed spectrally constant)\\
\codetab{iseed} & \codetab{col} & Seed for McICA random-number generator (double precision, default: 1, 2, 3...)\\
\codetab{pressure\_hl} & \codetab{col, half\_level} & Pressure at half levels (Pa) \\
\codetab{temperature\_hl}& \codetab{col, half\_level} & Temperature at half levels (K) \\
\codetab{q} or \codetab{h2o\_mmr} & \codetab{col, level} & Specific humidity (kg~kg$^{-1}$)\\
\codetab{h2o\_vmr} & \codetab{col, level} & Water vapour volume mixing ratio (mol~mol$^{-1}$)\\
\codetab{o3\_mmr}& \codetab{col, level} & Ozone mass mixing ratio (kg~kg$^{-1}$)\\
\codetab{o3\_vmr} & \codetab{col, level} & Ozone volume mixing ratio (mol~mol$^{-1}$),  used only if \codetab{o3\_mmr} not provided\\
\codetab{aerosol\_mmr} & \codetab{col, aer\_type, level} & Aerosol mass mixing ratio (kg~kg$^{-1}$)\\
\codetab{q\_liquid} & \codetab{col, level} & Liquid cloud mass mixing ratio (kg~kg$^{-1}$)\\
\codetab{q\_ice} & \codetab{col, level} & Ice cloud mass mixing ratio (kg~kg$^{-1}$)\\
\codetab{re\_liquid} & \codetab{col, level} & Liquid cloud effective radius (m)\\
\codetab{re\_ice} & \codetab{col, level} & Ice cloud effective radius (m)\\
\codetab{cloud\_fraction} & \codetab{col, level} & Cloud fraction\\
\codetab{overlap\_param} & \codetab{col, level\_interface} & Cloud overlap parameter (default: compute from decorrelation length of 2~km)\\
\codetab{fractional\_std} & \codetab{col, level} & Fractional standard deviation of cloud optical depth (default 0)\\
\codetab{inv\_cloud\_effective\_size} & \codetab{col, level} & Inverse of cloud effective horizontal size for SPARTACUS solver (m$^{-1}$)\\
\codetab{inv\_inhom\_effective\_size} & \codetab{col, level} & Inverse of effective horizontal size of cloud inhomogeneities, for SPARTACUS solver (m$^{-1}$) (default: same as \codetab{inv\_cloud\_effective\_size})\\
\codetab{inv\_cloud\_effective\_separation} & \codetab{col, level} & Alternative input to SPARTACUS if \codetab{inv\_cloud\_effective\_size} not present (m$^{-1}$)\\
\codetab{inv\_inhom\_effective\_separation} & \codetab{col, level} & Alternative input to SPARTACUS if \codetab{inv\_inhom\_effective\_size} not present (m$^{-1}$)\\
\hline
\end{longtable}
\end{center}

All the test data store input fields in order of increasing pressure,
i.e.\ starting at the top-of-atmosphere and working down to the
surface. The output data are then provided using the same
convention. If input data are provided in the opposite order then this
should be automatically detected and under the bonnet the order is
reversed before being passed to the radiation scheme. But if you use
this convention then please test the results carefully as this option
is not regularly tested. The variables describing cloud properties,
particularly sub-grid cloud struture, are defined in detail in section
\ref{sec:cloud_structure}.

The output netCDF file contains the typical set of variables listed in
Table \ref{tab:outvar}. Clear-sky fluxes (i.e.\ computed on the same
input profiles but in the absence of clouds) are provided if the
\code{do\_clear} namelist parameter is set to \code{true} (see section
\ref{sec:nam_radiation}). If you need diagnostic downward fluxes at
the surface for just a subset of the spectrum (e.g. ultraviolet or
photosynthetically active radiation) then they can be computed from
the \code{spectral\_flux\_dn\_*} variables, activated if namelist
variable \code{do\_surface\_sw\_spectral\_flux} is set to
\code{true}. In some contexts it is also useful to have fluxes in each
of the shortwave albedo or longwave emissivity spectral intervals.
These are named \code{canopy\_flux\_dn\_*} and are activated if
\code{do\_canopy\_fluxes\_sw} or \code{do\_canopy\_fluxes\_lw} are set
to \code{true}. Note that if you want atmospheric heating rates then
you will need to compute them yourself from the flux profiles.

\begin{center}
\tablesetup
\begin{longtable}{llLp{5.9cm}}%
\caption{\label{tab:outvar}Variables contained in the output netCDF
  file from \spsurf, where all fluxes (or irradiances) have units of
  W~m$^{-2}$. The \code{band\_sw} dimension has the same size as the
  number of shortwave bands in the gas-optics scheme.}\\
%
\hline
Variable & Dimensions & Description\\
\hline
\codetab{pressure\_hl} & \codetab{col, half\_level} & Pressure at half levels (Pa)\\
\codetab{flux\_up\_sw, flux\_dn\_sw} & \codetab{col, half\_level} & Up- and downwelling shortwave fluxes \\
\codetab{flux\_up\_sw\_clear, flux\_dn\_sw\_clear} & \codetab{col, half\_level} & Up- and downwelling clear-sky shortwave fluxes\\
\codetab{flux\_dn\_direct\_sw} & \codetab{col, half\_level} & Direct component of downwelling shortwave flux\\
\codetab{flux\_dn\_direct\_sw\_clear} & \codetab{col, half\_level} & Direct component of downwelling clear-sky shortwave flux\\
\codetab{flux\_up\_lw, flux\_dn\_lw} & \codetab{col, half\_level} & Up- and down-welling longwave fluxes \\
\codetab{flux\_up\_lw\_clear, flux\_dn\_lw\_clear} & \codetab{col, half\_level} & Up- and down-welling clear-sky longwave fluxes\\
\codetab{lw\_derivative} & \codetab{col, half\_level} & Derivative of upwelling longwave flux with respect to surface value \citep{Hogan+2015}\\
\codetab{spectral\_flux\_dn\_sw\_surf} & \codetab{col, band\_sw} & Downwelling surface shortwave flux in each band\\
\codetab{spectral\_flux\_dn\_direct\_sw\_surf} & \codetab{col, band\_sw} & Direct downwelling surface shortwave flux in each band\\
\codetab{spectral\_flux\_dn\_sw\_surf\_clear} & \codetab{col, band\_sw} & Clear-sky downwelling surface shortwave flux in each band\\
\codetab{spectral\_flux\_dn\_direct\_sw\_surf\_clear} & \codetab{col, band\_sw} & Clear-sky direct downwelling surface shortwave flux in each band\\
\codetab{canopy\_flux\_dn\_diffuse\_sw\_surf} & \codetab{col, sw\_albedo\_band} & Downwelling diffuse surface shortwave flux in each albedo interval \\
\codetab{canopy\_flux\_dn\_direct\_sw\_surf} & \codetab{col, sw\_albedo\_band} & Downwelling direct surface shortwave flux in each albedo interval \\
\codetab{canopy\_flux\_dn\_lw\_surf} & \codetab{col, lw\_emiss\_band} & Downwelling surface longwave flux in each emissivity interval \\
\codetab{cloud\_cover\_sw} & \codetab{col} & Total cloud cover diagnosed by shortwave solver\\
\codetab{cloud\_cover\_lw} & \codetab{col} & Total cloud cover diagnosed by longwave solver\\
\hline
\end{longtable}
\end{center}


\section{Configuring the radiation scheme}
\label{sec:nam_radiation}
The detailed settings of \spsurf\ are configured using the
\code{radiation} namelist in the namelist file provided as the first
command-line argument to the \code{ecrad} executable. The available
namelist parameters are listed in Table \ref{tab:nam_radiation}. One of
the most important is \code{directory\_name}, which provides the
absolute or relative path to the directory containing all the
configuration files. This is the \code{data} directory at the top
level of the \spsurf\ package. Note that the default values listed in
Table \ref{tab:nam_radiation} may differ in some cases from the values
used operationally in the IFS \cite[see Table 2 of][]{Hogan+2018}.

%\newcommand{\namedef}{3}{\code{#1} & #2 & #3\\}

%\begin{table}
\begin{center}
\tablesetup
%\begin{longtable}{lc>{\raggedright}p{5cm}>{\raggedright}p{5cm}}
%\begin{longtable}{lXLp{4cm}Lp{5.5cm}}
\begin{longtable}{lXLp{3.5cm}Lp{6cm}}
%
\caption{\label{tab:nam_radiation}Options for the \code{radiation}
  namelist that configures the radiation scheme. The type of each
  parameter can be inferred from its name: logicals begin with
  \code{do\_} or \code{use\_}, integers start with \code{i\_} or
  \code{n\_}, strings end with \code{\_name}, and all other parameters
  are real numbers.}\\
%
\hline
Parameter & Type & \textbf{Default value}, other values & Description\\
\hline
%\namedef{do_sw}{\codetab{true},\codetab{false}}{Run the shortwave scheme?}
\multicolumn{4}{l}{\emph{General}}\\
\codetab{directory\_name} & S & \textemph{.} & Directory containing netCDF configuration files \\
\codetab{do\_sw} & L & \codetabemph{true} & Compute shortwave fluxes?\\
\codetab{do\_lw} & L & \codetabemph{true} & Compute longwave fluxes?\\
\codetab{do\_sw\_direct} & L & \codetabemph{true} & Do direct shortwave fluxes? \\
\codetab{do\_clear} & L & \codetabemph{true} & Compute clear-sky fluxes? \\
\hline
\multicolumn{4}{l}{\emph{Gas and aerosol optics}}\\
\codetab{gas\_model\_name} & S & \textemph{RRTMG-IFS}, Monochromatic & Gas optics model \\
\codetab{use\_aerosols} & L & \codetabemph{false} & Do we represent aerosols? \\
\codetab{do\_lw\_aerosol\_scattering} & L & \codetabemph{true} & Do longwave aerosol scattering? \\
\codetab{n\_aerosol\_types} & I & & Number of aerosol types \\
\codetab{i\_aerosol\_type\_map} & I(:) & & Vector of integers that map from aerosol types to types in the netCDF aerosol optics file, where positive integers indexe hydrophobic types, negative integers index hydrophilic types and zero indicates a type should be ignored\\
\codetab{aerosol\_optics\_override\_file\_name} & S & & Path to an alternative aerosol optics file\\
\hline
\multicolumn{4}{l}{\emph{Monochromatic scheme}}\\
\codetab{mono\_lw\_wavelength} & R & $\bf -$\textbf{1.0} & Wavelength of longwave radiation, or if negative, a broadband calculation will be performed\\
\codetab{mono\_lw\_total\_od} & R & \textemph{0.0} & Zenith longwave optical depth of clear-sky atmosphere\\
\codetab{mono\_sw\_total\_od} & R & \textemph{0.0} & Zenith shortwave optical depth of clear-sky atmosphere\\
\codetab{mono\_lw\_single\_scattering\_albedo} & R & \textemph{0.538} & Longwave cloud single scattering albedo\\
\codetab{mono\_sw\_single\_scattering\_albedo} & R & \textemph{0.999999} & Shortwave cloud single scattering albedo\\
\codetab{mono\_lw\_asymmetry\_factor} & R & \textemph{0.925} & Longwave cloud asymmetry factor\\
\codetab{mono\_sw\_asymmetry\_factor} & R & \textemph{0.86} & Shortwave cloud asymmetry factor\\
\hline
\multicolumn{4}{l}{\emph{Cloud optics}}\\
\codetab{liquid\_model\_name} & S & \textemph{SOCRATES}, Slingo, Monochromatic & Liquid optics model, including the scheme in the SOCRATES radiation scheme and the older scheme of \cite{Slingo1989} \\
\codetab{ice\_model\_name} & S & \textemph{Fu-IFS}, Baran2016, Yi, Monochromatic & Ice optics model, including the schemes of \cite{Fu1996}, \cite{Fu+1998}, \cite{Baran+2016} and \cite{Yi+2013} \\
\codetab{do\_lw\_cloud\_scattering} & L & \codetabemph{true} & Do longwave cloud scattering? \\
\codetab{do\_fu\_lw\_ice\_optics\_bug} & L & \codetabemph{false} & Reproduce bug in McRad implementation of Fu ice optics \citep{Hogan+2016}? \\
\codetab{liq\_optics\_override\_file\_name} & S & & Path to alternative liquid optics file name\\
\codetab{ice\_optics\_override\_file\_name} & S & & Path to alternative ice optics file name\\
\hline
\multicolumn{4}{l}{\emph{Solver}}\\
\codetab{sw\_solver\_name} & S & Cloudless, Homogeneous, \textemph{McICA}, Tripleclouds, SPARTACUS & Shortwave solver; note that the homogeneous solver assumes cloud fills the gridbox horizontally (so ignores cloud fraction) while the cloudless solver ignores clouds completely \\
\codetab{lw\_solver\_name} & S & Cloudless, Homogeneous, \textemph{McICA}, Tripleclouds, SPARTACUS & Longwave solver \\
\codetab{overlap\_scheme\_name} & S & Max-Ran, \textemph{Exp-Ran}, Exp-Exp & Cloud overlap scheme; note that SPARTACUS and Tripleclouds only work with the Exp-Ran overlap scheme \\
\codetab{use\_beta\_overlap} & L & \codetabemph{false} & Use \cite{Shonk+2010} `$\beta$' overlap parameter definition, rather than default `$\alpha$'? \\
\codetab{cloud\_inhom\_decorr\_scaling} & R & \textemph{0.5} & Ratio of overlap decorrelation lengths for cloud inhomogeneities and boundaries \\
\codetab{cloud\_fraction\_threshold} & R & $\bf 10^{-6}$ & Ignore clouds with fraction below this \\
\codetab{cloud\_mixing\_ratio\_threshold} & R & $\bf 10^{-9}$ & Ignore clouds with total mixing ratio below this \\
\codetab{cloud\_pdf\_shape\_name} & S & \textemph{Gamma}, Lognormal & Shape of cloud water PDF\\
\codetab{cloud\_pdf\_override\_file\_name} & S & & Name of netCDF file of alternative cloud PDF look-up table\\
\codetab{do\_sw\_delta\_scaling\_with\_gases} & L & \codetabemph{false} & Apply delta-Eddington scaling to particle-gas mixture, rather than particles only \citep[see][]{Hogan+2018}\\
\hline
\multicolumn{4}{l}{\emph{SPARTACUS solver (these parameters have no effect for other solvers)}}\\
\codetab{do\_3d\_effects} & L & \codetabemph{true} & Represent cloud edge effects when SPARTACUS solver selected; note that this option does not affect entrapment, which is also a 3D effect\\
\codetab{n\_regions} & I & 2, \textemph{3} & Number of regions, where one is clear sky and one or two are cloud (the Tripleclouds solver always assumes three regions regardless of this parameter) \\
\codetab{do\_lw\_side\_emissivity} & L & \codetabemph{true} & Represent effective emissivity of the side of clouds \citep{Schafer+2016}\\
\codetab{sw\_entrapment\_name} & S & Zero, Edge-only, \textemph{Explicit}, Non-fractal, Maximum & Entrapment model \citep{Hogan+2019}; note that the behaviour in ecRad version 1.0.1 was `Maximum' entrapment\\
\codetab{do\_3d\_lw\_multilayer\_effects} & L & \codetabemph{false} & Maximum entrapment for longwave radiation?\\
\codetab{max\_3d\_transfer\_rate} & R & \textemph{10.0} & Maximum rate of lateral exchange between regions in one layer, for stability of matrix exponential (where the default means that as little as $e^{-10}$ of the radiation could remain in a region)\\
\codetab{max\_gas\_od\_3d} & R & \textemph{8.0} & 3D effects ignored for spectral intervals where gas optical depth of a layer exceeds this, for stability\\
\codetab{max\_cloud\_od} & R & \textemph{16.0} & Maximum in-cloud optical depth, for stability\\
\codetab{use\_expm\_everywhere} & L & \codetabemph{false} & Use matrix-exponential method even when 3D effects not important, such as clear-sky layers and parts of the spectrum where the gas optical depth is large?\\
\codetab{clear\_to\_thick\_fraction} & R & \textemph{0.0} & Fraction of cloud edge interfacing directly to the most optically thick cloudy region\\
\codetab{overhead\_sun\_factor} & R & \textemph{0.0} & Minimum tan-squared of solar zenith angle to allow some `direct' radiation from overhead sun to pass through cloud sides \citep[0.06 used by][]{Hogan+2016}\\
\codetab{overhang\_factor} & R & \textemph{0.0} & A detail of the entrapment representation described by \cite{Hogan+2019}\\
\hline
%\multicolumn{4}{l}{\emph{Surface (under development)}}\\
%\codetab{use\_canopy\_full\_spectrum\_sw} & L & \codetabemph{false} & Perform canopy shortwave radiative transfer at full atmospheric spectral resolution \\
%\codetab{use\_canopy\_full\_spectrum\_lw} & L & \codetabemph{false} & Perform canopy longwave radiative transfer at full atmospheric spectral resolution \\
\multicolumn{4}{l}{\emph{Surface}}\\
\codetab{do\_nearest\_spectral\_sw\_albedo} & L & \textemph{true} & Surface shortwave albedos may be supplied in their own spectral intervals: do we select the nearest to each band of the gas optics scheme, rather than using a weighted average? \\
\codetab{do\_nearest\_spectral\_lw\_emiss} & L & \textemph{true} & ...likewise but for surface longwave emissivity\\
\codetab{sw\_albedo\_wavelength\_bound} & R & & Vector of the wavelength bounds (m) delimiting the shortwave albedo intervals\\
\codetab{lw\_emiss\_wavelength\_bound} & R & & Vector of the wavelength bounds (m) delimiting the longwave emissivity intervals\\
\codetab{i\_sw\_albedo\_index} & I & & Vector of indices mapping albedos to wavelength intervals\\
\codetab{i\_lw\_emiss\_index} & I & & Vector of indices mapping emissivities to wavelength intervals\\
\hline
\multicolumn{4}{l}{\emph{Diagnostics}}\\
\codetab{iverbosesetup} & I & 0, 1, 2, \textemph{3}, 4, 5 & Verbosity in setup, where 1=warning, 2=info, 3=progress, 4=detailed, 5=debug \\
\codetab{iverbose} & I & 0, \textemph{1}, 2, 3, 4, 5 & Verbosity in execution\\
\codetab{do\_save\_spectral\_flux} & L & \codetabemph{false} & Save flux profiles in each band?\\
\codetab{do\_save\_gpoint\_flux} & L & \codetabemph{false} & Save flux profiles in each g-point?\\
\codetab{do\_surface\_sw\_spectral\_flux} & L & \codetabemph{true} & Save surface shortwave fluxes in each band for subsequent diagnostics?\\
\codetab{do\_lw\_derivatives} & L & \codetabemph{false} & Compute derivatives for \cite{Hogan+2015} approximate updates?\\
\codetab{do\_save\_radiative\_properties} & L & \codetabemph{false} & Write intermediate netCDF file(s) of properties sent to solver (\codetab{radiative\_properties*.nc})?\\
\codetab{do\_canopy\_fluxes\_sw} & L & \codetabemph{false} & Save surface shortwave fluxes in each albedo interval\\
\codetab{do\_canopy\_fluxes\_lw} & L & \codetabemph{false} & Save surface longwave fluxes in each emissivity interval\\
\hline
% To be added:
%     &  do_canopy_gases_sw, do_canopy_gases_lw, 
\end{longtable}
\end{center}
%\end{table}

Several of the entries in Table \ref{tab:nam_radiation} are configured
with vectors of numbers, which deserves further explanation. As shown
in Table \ref{tab:invar}, aerosols are provided to \spsurf\ in the form
of the mass mixing ratios of a number of different aerosol types. The
optical properties of an arbitrary number of hydrophilic and
hydrophobic aerosol types is provided in a netCDF file, for example
\code{data/aerosol\_ifs\_rrtm\_45R2.nc} in the \spsurf\ package. The
mapping between the input aerosol concentrations and the aerosol types
in the optical-property file may be specified in the \code{radiation}
namelist. The \code{n\_aerosol\_types} parameter specifies the number
of aerosol concentrations to be provided, with a value of zero having
the effect of deactivating aerosols. \code{i\_aerosol\_type\_map} is a
vector of integers of length \code{n\_aerosol\_types} indicating which
aerosol type to select from the optical-property file. Negative
numbers select hydrophilic types, whose optical properties vary with
relative humidity, while postitive numbers select hydrophobic
types. Zero indicates that an input aerosol type is to be ignored. As
an example, the IFS settings (in the \code{test/ifs} directory) are
specified with:
%

\begin{lstlisting}
 aerosol_optics_override_file_name = 'aerosol_ifs_rrtm_46R1_with_NI_AM.nc'
 n_aerosol_types = 12
 i_aerosol_type_map = -1, -2, -3, 1, 2, 3, -4, 10, 11, 11, -5, 14
\end{lstlisting}
%\normalsize
%
When \spsurf\ is run, the output printed to the terminal includes a
description of the aerosol mapping.

A similar mechanism is used to describe how spectral intervals of the
input \code{sw\_albedo} and \code{lw\_emissivity} should be
interpretted.  This is best explained by considering the configuration
of the IFS in Cycle 47R1, which is described by the following namelist
variables:
%

\footnotesize
\begin{lstlisting}
 sw_albedo_wavelength_bound(1:5) = 0.25e-6, 0.44e-6, 0.69e-6, 1.19e-6, 2.38e-6
 i_sw_albedo_index(1:6) = 1,2,3,4,5,6
 do_nearest_spectral_sw_albedo = false
 lw_emiss_wavelength_bound(1:2) = 8.0e-6, 13.0e-6
 i_lw_emiss_index(1:3) = 1,2,1
 do_nearest_spectral_lw_emiss = true
\end{lstlisting}
\normalsize
%
The IFS describes surface albedo in six spectral intervals. The vector
\code{sw\_albedo\_wavelength\_bounds} here provides the wavelengths,
in metres, of the five boundaries between these intervals, where the
first interval is taken to include all wavelengths shorter than the
first value (in this case 0.25~$\mu$m) and the last includes all
wavelengths longer than the last value (in this case 2.38~$\mu$m). The
vector \code{i\_sw\_albedo\_index} specifies which of the elements of
the input \code{sw\_albedo} field should be used in each of the six
spectral intervals.  Surface emissivity is described similarly: there
are three spectral intervals specified by the two boundaries in
\code{lw\_emiss\_wavelength\_bound}. The corresponding vector
\code{i\_lw\_emiss\_index} contains two occurrences of the index
\code{1}, indicating that the first element of \code{lw\_emissivity}
is used both for wavelengths smaller than 8~$\mu$m and wavelengths
larger than 13~$\mu$m (i.e.\ outside the infrared atmospheric
window). The second element is then used for wavelengths between these
two boundaries. Thus even though there are three spectral intervals,
only two elements are needed in \code{lw\_emissivity}.  Finally, the
logicals \code{do\_nearest\_spectral\_sw\_albedo} and
\code{do\_nearest\_spectral\_lw\_emiss} specify whether the bands of
the gas optics scheme used in \spsurf\ will use a single value of
albedo or emissivity from the input fields (chosen to be the spectral
interval with the largest overlap in wavenumber space with each band
of the gas-optics scheme), or whether they will weight the spectral
intervals by their overlap with each band of the gas-optics
scheme. The mapping from spectral interval to band is printed on
standard output when \spsurf\ is run, as shown in the example in
section \ref{sec:checking}.


\section{Configuring the offline package}
\label{sec:nam_radiation_config}
In addition to the namelist parameters described in section
\ref{sec:nam_radiation} an additional set of parameters are available
in the \code{radiation\_config} namelist that are specific to the
offline version of \spsurf\ and are listed in Table
\ref{tab:nam_radiation_config}. In general if these parameters are
present in the namelist then they will override the corresponding
variable provided in the input file.

\begin{center}
\tablesetup
\begin{longtable}{ll}
%
\caption{\label{tab:nam_radiation_config}Options for the
  \code{radiation\_config} namelist that configures additional aspects
  of the offline radiation scheme. All entries must be scalars. If an
  override parameter is present then it need not be included in the
  input file. The cloud effective sizes (used by the SPARTACUS solver)
  may be specified for low, middle and high clouds according to the
  cloud layer pressure $p$ and the surface pressure $p_0$.}\\
%
\hline
Parameter & Description\\
\hline
\multicolumn{2}{l}{\emph{Execution control}}\\
\codetab{nrepeat}  & Number of times to repeat, for benchmarking\\
\codetab{istartcol} & Start at specified input column (1 based)\\
\codetab{iendcol} & End at specified input column (1 based)\\
\codetab{iverbose} & Verbosity in offline setup (default 2)\\
\codetab{do\_parallel} & Use OpenMP parallelism? (default \codetab{true})\\
\codetab{nblocksize} & Number of columns per block when using OpenMP\\
\codetab{do\_save\_inputs} & Sanity check: save input variables in \codetab{inputs.nc}\\
\codetab{do\_correct\_unphysical\_inputs} & If input variables out of physical bounds, correct them and issue a warning\\
\codetab{vmr\_suffix\_str} & Suffix for variables containing volume mixing ratios (default `\code{\_vmr}')\\
\hline
\multicolumn{2}{l}{\emph{Override input variables}}\\
\codetab{solar\_irradiance\_override} & Override solar irradiance (W~m$^{-2}$)\\
\codetab{skin\_temperature} & Override skin temperature (K)\\
\codetab{cos\_solar\_zenith\_angle} & Override cosine of solar zenith angle\\
\codetab{sw\_albedo} & Override shortwave albedo\\
\codetab{lw\_emissivity} & Override longwave emissivity\\
\codetab{fractional\_std} & Override cloud optical depth fractional standard deviation\\
\codetab{overlap\_decorr\_length} & Override cloud overlap decorrelation length (m)\\
\codetab{inv\_effective\_size} & Override inverse of cloud effective size (m$^{-1}$)\\
\codetab{low\_inv\_effective\_size} & ...for low clouds ($p>0.8p_0$, where $p$ is pressure and $p_0$ surface pressure)\\
\codetab{middle\_inv\_effective\_size} & ...for mid-level clouds ($0.45p_0< p\le 0.8p_0$)\\
\codetab{high\_inv\_effective\_size} & ...for high clouds ($p\le 0.45p_0$)\\
\hline
\multicolumn{2}{l}{\emph{Scale input variables}}\\
\codetab{q\_liquid\_scaling} & Scaling for liquid water mixing ratio\\
\codetab{q\_ice\_scaling} & Scaling for ice water mixing ratio\\
\codetab{cloud\_fraction\_scaling} & Scaling for cloud fraction (capped at 1)\\
\codetab{overlap\_decorr\_length\_scaling} & Scaling for cloud overlap decorrelation length\\
\codetab{effective\_size\_scaling} & Scaling for cloud effective size\\
\codetab{h2o\_scaling, co2\_scaling...} & Scaling for specific humidity and carbon dioxide; equivalents available for\\
& \code{o3}, \code{co}, \code{ch4},
\code{n2o}, \code{o2}, \code{cfc11}, \code{cfc12}, \code{hcfc22} and
\code{ccl4}\\
\hline
\multicolumn{2}{l}{\emph{Parameterize input variables}}\\
\codetab{cloud\_inhom\_separation\_factor} & Set inhomogeneity separation scale to be this multiplied by cloud separation scale\\
\codetab{cloud\_separation\_scale\_surface} & Surface cloud separation scale in pressure-dependent parameterization \\
\codetab{cloud\_separation\_scale\_toa} & Top-of-atmosphere cloud separation scale in pressure-dependent parameterization\\
\codetab{cloud\_separation\_scale\_power} & Power in cloud separation scale parameterization \\
\hline
\end{longtable}
\end{center}

\section{Describing cloud structure}
\label{sec:cloud_structure}
Probably more than any other 1D radiation scheme, \spsurf\ allows the
user to define in detail the statistical properties of the sub-grid
cloud distribution, and in this section the relevant variables and
namelist parameters are explained in more detail.  In an operational
context most of these variables need to be parameterized, but in
developing new solvers we need to perform explicit radiation
calculations on realistic high resolution 3D cloud fields, and compare
them to \spsurf\ simulations in which the profiles of these variables
have been extracted from the 3D cloud fields.  This has been done by
\cite{Schafer+2016}, \cite{Hogan+2016} and \cite{Hogan+2019}. Explicit
radiation calculations on a 3D cloud field can either be performed
using the Independent Column Approximation (ICA) and compared to
\spsurf's McICA or Tripleclouds solvers, or using a fully 3D solver
(e.g.\ Monte Carlo) and comparing it to \spsurf's SPARTACUS solver.
Note that \spsurf\ can itself perform ICA calculations on 3D cloud
fields, by flattening the two horizontal dimensions of a 3D dataset
into a single `column' dimension, and using the \spsurf's `Homogeneous'
solver in which any cloud is assumed to homogeneously fill each of the
narrow columns (so cloud fraction is not used as it is implicitly
taken to be 0 or 1).

The input variables describing the profile of cloud properties are
given in the lower half of Table \ref{tab:invar}. The most basic are
the liquid and ice mass mixing ratios (\code{q\_liquid} and
\code{q\_ice}), which are gridbox-mean quantities, and the
corresponding effective radii (\code{re\_liquid} and
\code{re\_ice}). Effective radius is assumed to be horizontally
constant within a gridbox, even if the water content varies.  For all
cloud optics models, effective radius is defined as
%
\begin{align}
r_{e,\mathrm{liq}}&=\frac{3\mathrm{LWC}}{4\rho_\mathrm{liq}A_\mathrm{liq}};\\
r_{e,\mathrm{ice}}&=\frac{3\mathrm{IWC}}{4\rho_\mathrm{ice}A_\mathrm{ice}},
\end{align}
%
where LWC and IWC are the liquid and ice water contents (i.e.\ the
mass mixing ratios multiplied by the air density), $\rho_\mathrm{liq}$
and $\rho_\mathrm{ice}$ are the densities of liquid water and solid
ice, and $A_\mathrm{liq}$ and $A_\mathrm{ice}$ are the total projected
cross-sectional areas of liquid droplets and ice particles per unit
volume of air (so units of m$^{-1}$). 

Cloud fraction is simply the fractional horizontal area of a given
model layer that contains cloud.  The layers are assumed to be thin
enough that cloud fraction is constant with height within a layer,
i.e.\ cloud fraction by volume is equal to cloud fraction by area. The
horizontal variability of cloud water content in a layer is specified
by the fractional standard deviation (\code{fractional\_std}), defined
as the standard deviation of the in-cloud water content, divided by
the in-cloud mean water content. The in-cloud mean water content is
the gridbox-mean water content divided by cloud fraction. Note that
since effective radius is assumed constant across a gridbox, cloud
optical depth is proportional to water path and so
\code{fractional\_std} can also be thought of as the horizontal
fractional standard deviation of cloud optical depth. Moreover,
\spsurf\ assumes that horizontal variations of liquid and ice water
content are perfectly correlated.  As shown in Table
\ref{tab:nam_radiation_config}, fractional standard deviation can be
overriden through a namelist parameter; for example, in the IFS this
value is set to 1.

Cloud overlap is needed by the Exp-Ran and Exp-Exp overlap schemes,
and is specified at the interface (or half-level) between each layer
by \code{overlap\_param}, the overlap parameter as defined by
\cite{Hogan&2000}.  To compute this at half-level $i+1/2$ of a
high-resolution 3D cloud field, you need the cloud fractions in the
upper and lower lower layers, $c_{i}$ and $c_{i+1}$, and the combined
cloud cover of the cloud in these two layers, $C$.  Then from Eqs.\ 1,
2 and 4 of \cite{Hogan&2000} you can compute the overlap parameter:
%
\begin{equation}
  \alpha_{i+1/2}=\frac{C_\mathrm{rand}-C}{C_\mathrm{max}-C},
\end{equation}
%
where the combined cloud covers that would be obtained from the random
and maximum overlap assumptions are
%
\begin{align}
  C_\mathrm{rand}&=c_i+c_{i+1}-c_ic_{i+1};\\
  C_\mathrm{max}&=\mathrm{max}(c_i,c_{i+1}).
\end{align}
%
Alternatively, cloud overlap can be parameterized as in most
atmospheric models in terms of an overlap decorrelation length as
shown in Table \ref{tab:nam_radiation_config}, which implements Eq.\ 5
of \cite{Hogan&2000}. In addition to describing how cloud boundaries
overlap, \spsurf\ needs to know how sub-grid cloud inhomogeneities are
vertically correlated. This cannot be specified at each layer, but is
rather specified via the namelist variable
\code{cloud\_inhom\_decorr\_scaling} in Table \ref{tab:nam_radiation},
which gives the ratio of the decorrelation lengths for cloud
inhomogeneities and cloud boundaries. The default value of 0.5 was
obtained from observations of ice clouds by \cite{Hogan+2003}.

The variables and parameters above are all used by the McICA and
Tripleclouds solvers to represent cloud properties relevant for 1D
radiative transfer.  In order to use the SPARTACUS solver to represent
3D radiative effects, we also need a means to specify the
\emph{normalized cloud perimeter length,} $L$, in each model layer.
If we imagine a horizontal slice through the sub-grid cloud field,
then $L$ is the total cloud perimeter length divided by the area of
the domain, with units of inverse metres. This variable is not
provided to SPARTACUS directly, since it tends to be strongly
dependent on the cloud fraction.  Rather we specify either the
\emph{cloud effective size,} $C_S$, or the \emph{cloud effective
  separation,} $C_X$, which tend to be less dependent on cloud fraction.
Normalized perimeter length is related to the former via Eq.\ 29 of
\cite{Hogan+2019}:
%
\begin{equation}
  L=4c(1-c)/C_S,\label{eq:S}
\end{equation}
%
and to the latter via \cite[]{Fielding+2020}
%
\begin{equation}
  L=4\left[c(1-c)\right]^{1/2}/C_X,\label{eq:X}
\end{equation}
%
where $c$ is the cloud fraction. The variables $1/C_S$ and $1/C_X$ may
be specified directly in the input file as
\code{inv\_cloud\_effective\_size} and
\code{inv\_cloud\_effective\_separation}, respectively. If both are
present then the former will take precedence. The reason that
reciprocals are provided is that then a value of zero (corresponding
to $C_S$ or $C_X$ of infinity) indicates no 3D effects are to be simulated
in a particular layer. If you have a high resolution cloud scene and
you wish to wish to run SPARTACUS on it then you need to compute the
perimeter length from it (e.g.\ use a contouring function on a field
containing 0 for clear sky and 1 for cloud, and then compute the
length of the 0.5 contour), and knowing also cloud fraction you can
invert (\ref{eq:S}) or (\ref{eq:X}).

In the context of an atmospheric model, we recommend that $C_X$ is
parameterized using the namelist parameters at the bottom of Table
\ref{tab:nam_radiation_config} scheme with the values of
\cite{Fielding+2020}:
%
\begin{lstlisting}
 cloud_separation_scale_toa     = 14000.0, ! Value of C_X at top-of-atmosphere (m)
 cloud_separation_scale_surface = 2500.0,  ! Value of C_X at surface (m)
 cloud_separation_scale_power   = 3.5,     ! Describes pressure dependence of C_X
 cloud_inhom_separation_factor  = 0.75     ! Defines size of cloud inhomogeneities
\end{lstlisting}
%
These numbers are used in the namelist in the \code{test/ifs}
case. Note that the first number shown here, $C_X^\mathrm{TOA}$, is
valid for a model with a horizontal grid spacing of around 100~km, but
this parameter was found by \cite{Fielding+2020} to be dependent on
horizontal grid spacing $\Delta x$ in a way that can be fitted with
%
\begin{equation}
C_X^\mathrm{TOA}=1.62\,\Delta x^{0.47},
\end{equation}
%
where both $C_X^\mathrm{TOA}$ and $\Delta x$ are in km. The surface
value of $C_X$ can be assumed to be 2.5~km for all model
resolutions.

% ADD DEFINITION OF INHOM SEP FACTOR

\section{Checking the configuration}
\label{sec:checking}
When \code{ecrad} is run, it outputs to the screen a summary of the
configuration options, the files read and written and details of the
aerosol mapping. This can be used to check that \spsurf\ has been
configured as intended.  The following is an example from the default
test in the \code{test/ifs} directory, in the case of
\code{iverbosesetup=2} and \code{iverbose=1} in the \code{radiation}
namelist:

\footnotesize
\begin{verbatim}
-------------------------- OFFLINE ECRAD RADIATION SCHEME --------------------------
Copyright (C) 2014-2019 European Centre for Medium-Range Weather Forecasts
Contact: Robin Hogan (r.j.hogan@ecmwf.int)
Floating-point precision: double
General settings:
  Data files expected in "../../data"
  Clear-sky calculations are ON                              (do_clear=T)
  Saving intermediate radiative properties OFF               (do_save_radiative_properties=F)
  Saving spectral flux profiles ON                           (do_save_spectral_flux=T)
  Gas model is "RRTMG-IFS"                                   (i_gas_model=1)
  Aerosols are ON                                            (use_aerosols=T)
  Clouds are ON                                              (do_clouds=T)
Surface settings:
  Saving surface shortwave spectral fluxes OFF               (do_surface_sw_spectral_flux=F)
  Saving surface shortwave fluxes in abledo bands ON         (do_canopy_fluxes_sw=T)
  Saving surface longwave fluxes in emissivity bands ON      (do_canopy_fluxes_lw=T)
  Longwave derivative calculation is ON                      (do_lw_derivatives=T)
  Nearest-neighbour spectral albedo mapping OFF              (do_nearest_spectral_sw_albedo=F)
  Nearest-neighbour spectral emissivity mapping ON           (do_nearest_spectral_lw_emiss=T)
Cloud settings:
  Cloud fraction threshold = .100E-05                        (cloud_fraction_threshold)
  Cloud mixing-ratio threshold = .100E-08                    (cloud_mixing_ratio_threshold)
  Liquid optics scheme is "SOCRATES"                         (i_liq_model=2)
  Ice optics scheme is "Fu-IFS"                              (i_ice_model=2)
  Longwave ice optics bug in Fu scheme is OFF                (do_fu_lw_ice_optics_bug=F)
  Cloud overlap scheme is "Exp-Exp"                          (i_overlap_scheme=2)
  Use "beta" overlap parameter is OFF                        (use_beta_overlap=F)
  Cloud PDF shape is "Gamma"                                 (i_cloud_pdf_shape=1)
  Cloud inhom decorrelation scaling = .500                   (cloud_inhom_decorr_scaling)
Solver settings:
  Shortwave solver is "McICA"                                (i_solver_sw=2)
  Shortwave delta scaling after merge with gases OFF         (do_sw_delta_scaling_with_gases=F)
  Longwave solver is "McICA"                                 (i_solver_lw=2)
  Longwave cloud scattering is ON                            (do_lw_cloud_scattering=T)
  Longwave aerosol scattering is OFF                         (do_lw_aerosol_scattering=F)
Warning: turning on do_surface_sw_spectral_flux as required by do_canopy_fluxes_sw
Reading ../../data/RADRRTM
Reading ../../data/RADSRTM
Weighting of 6 albedo values in 14 shortwave bands (wavenumber ranges in cm-1):
  2600 to  3250: 0.00 0.00 0.00 0.00 0.00 1.00
  3250 to  4000: 0.00 0.00 0.00 0.00 0.00 1.00
  4000 to  4650: 0.00 0.00 0.00 0.00 0.69 0.31
  4650 to  5150: 0.00 0.00 0.00 0.00 1.00 0.00
  5150 to  6150: 0.00 0.00 0.00 0.00 1.00 0.00
  6150 to  7700: 0.00 0.00 0.00 0.00 1.00 0.00
  7700 to  8050: 0.00 0.00 0.00 0.00 1.00 0.00
  8050 to 12850: 0.00 0.00 0.00 0.93 0.07 0.00
 12850 to 16000: 0.00 0.00 0.48 0.52 0.00 0.00
 16000 to 22650: 0.00 0.00 1.00 0.00 0.00 0.00
 22650 to 29000: 0.00 0.99 0.01 0.00 0.00 0.00
 29000 to 38000: 0.00 1.00 0.00 0.00 0.00 0.00
 38000 to 50000: 0.83 0.17 0.00 0.00 0.00 0.00
   820 to  2600: 0.00 0.00 0.00 0.00 0.00 1.00
Mapping from 16 longwave bands to emissivity intervals: 1 1 1 1 1 2 2 2 1 1 1 1 1 1 1 1
Reading NetCDF file ../../data/socrates_droplet_scattering_rrtm.nc
Reading NetCDF file ../../data/fu_ice_scattering_rrtm.nc
Reading NetCDF file ../../data/aerosol_ifs_rrtm_46R1_with_NI_AM.nc
Aerosol mapping:
   1 -> hydrophilic type 1: Sea salt, bin 1, 0.03-0.5 micron, OPAC
   2 -> hydrophilic type 2: Sea salt, bin 2, 0.50-5.0 micron, OPAC
   3 -> hydrophilic type 3: Sea salt, bin 3, 5.0-20.0 micron, OPAC
   4 -> hydrophobic type 1: Desert dust, bin 1, 0.03-0.55 micron, (SW) Dubovik et al. 2002...
   5 -> hydrophobic type 2: Desert dust, bin 2, 0.55-0.90 micron, (SW) Dubovik et al. 2002...
   6 -> hydrophobic type 3: Desert dust, bin 3, 0.90-20.0 micron, (SW) Dubovik et al. 2002...
   7 -> hydrophilic type 4: Hydrophilic organic matter, OPAC
   8 -> hydrophobic type 10: Hydrophobic organic matter, OPAC (hydrophilic at RH=20%)
   9 -> hydrophobic type 11: Black carbon, OPAC
  10 -> hydrophobic type 11: Black carbon, OPAC
  11 -> hydrophilic type 5: Ammonium sulfate (for sulfate), GACP Lacis et al https://gacp...
  12 -> hydrophobic type 14: Stratospheric sulfate (hydrophilic ammonium sulfate at RH 20%-30%)
Reading NetCDF file ../../data/mcica_gamma.nc
Reading NetCDF file ecrad_meridian.nc
  Warning: variable co_vmr not found
  Warning: variable no2_vmr not found
Writing NetCDF file inputs.nc
Performing radiative transfer calculations
Writing NetCDF file ecrad_meridian_default_out.nc
------------------------------------------------------------------------------------
\end{verbatim}
\normalsize

%\section{Debugging}
%DEBUG=1
%radiative property file
%physical checks

%\chapter{Incorporating \spsurf\ into another program}
%\label{ch:api}

\spsurf\ can be called within a larger program, and indeed it has been
incorporated into several atmospheric models (the IFS, Meso-NH and
ICON).
%  It is designed to have a clean interface to facilitate
%this. 
Pending a full description here of how to do this, see the
\code{ifs/radiation\_setup.F90} in the \spsurf\ package to see how it
is configured in the IFS, and \code{ifs/radiation\_scheme.F90} for how
it is run.

%\section{Setup routine}

When calling \spsurf\ from within a model, the parameters listed in
Table \ref{tab:nam_radiation} are members of the \code{config\_type}
structure, and may be modified within the code at the appropriate
place in the configuration stage.  The exception is in the case of
strings, which are prefixed by \code{\_name} in the namelist.  In the
\code{config\_type} structure there are equivalent integers to express
these parameters, which can be changed using the named constants listed
in Table \ref{tab:named_constants}.

\begin{center}
\tablesetup
\begin{longtable}{>{\ttfamily}lMp{0.7\textwidth}}
\caption{\label{tab:named_constants}Integers in the
  \code{config\_type} structure that represents the strings in Table
  \ref{tab:nam_radiation}, where a namelist parameter named
  \code{*\_name} would be named \code{i\_*} here.}\\
%
\hline
\normalfont Variable in \codetab{config\_type} & \normalfont Available named constants, \textemph{default}\\
\hline
i\_overlap\_scheme &
IOverlapMaximumRandom, \codetabemph{IOverlapExponentialRandom}, IOverlapExponential\\
i\_solver\_sw, i\_solver\_lw &
ISolverCloudless, ISolverHomogeneous, \codetabemph{ISolverMcICA}, ISolverSpartacus, ISolverTripleclouds\\
i\_3d\_sw\_entrapment &
IEntrapmentZero, IEntrapmentEdgeOnly, \codetabemph{IEntrapmentExplicit}, IEntrapmentExplicitNonFractal, IEntrapmentMaximum\\
i\_gas\_model &
IGasModelMonochromatic, \codetabemph{IGasModelIFSRRTMG}\\
i\_liq\_model &
ILiquidModelMonochromatic, \codetabemph{ILiquidModelSOCRATES}, ILiquidModelSlingo\\
i\_ice\_model &
IIceModelMonochromatic,  \codetabemph{IIceModelFu}, IIceModelBaran2016, IIceModelYi\\
i\_cloud\_pdf\_shape &
\codetabemph{IPdfShapeGamma}, IPdfShapeLognormal\\
\hline
\end{longtable}
\end{center}

%\section{Compute routine}


\iffalse

\chapter{Code structure}
\label{ch:structure}

\section{Objects}


\chapter{Scientific documentation}
\label{ch:science}

\section{Gas optics}

\section{Aerosol optics}

\section{Cloud optics}

\section{Solvers}
\fi

\begin{thebibliography}{00}
\markright{References}
%
\harvarditem{Hogan et~al.}{2018}{Hogan+2018}Hogan, R. J., T. Quaife
and R. Braghiere, 2018: Fast matrix treatment of 3-D radiative
transfer in vegetation canopies: SPARTACUS-Vegetation
1.1. \textit{Geosci.\ Model Dev.,} \textbf{11,} 339-350.
%
\harvarditem{Hogan}{2019}{Hogan2019}Hogan, R. J., 2019: Flexible
treatment of radiative transfer in complex urban canopies for use in
weather and climate models. \textit{Boundary-Layer Meteorol.,}
\textbf{173,} 53-78.

%\harvarditem{Fielding~et~al.}{2020}{Fielding+2020}Fielding, M. D.,
%S. A. K. Sch\"afer, R. J. Hogan and R. M. Forbes, 2020: Encapsulating
%cloud geometry for 3D radiative transfer and cloud turbulent mixing
%parameterizations. \emph{To be submitted to Q. J. R. Meteorol.\ Soc.}
%
%\harvarditem{Hogan and Bozzo}{2018}{Hogan+2018}Hogan, R. J., and
%A. Bozzo, 2018: A flexible radiation scheme for the ECMWF
%model. \textit{J. Adv.\ Model.\ Earth Syst.,} \textbf{10,}
%doi:10.1029/2018MS001364.
%
\harvarditem{Hogan~et~al.}{2016}{Hogan+2016}Hogan, R. J.,
S. A. K. Sch\"afer, C. Klinger, J.-C. Chiu and B. Mayer, 2016:
Representing 3D cloud-radiation effects in two-stream schemes:
2. Matrix formulation and broadband
evaluation. \textit{J. Geophys.\ Res.,} \textbf{121,} 8583--8599.
%
%\harvarditem{Hogan~et~al.}{2019}{Hogan+2019}Hogan, R. J.,
%M. D. Fielding, H. W. Barker, N. Villefranque and S. A. K. Sch\"afer,
%2019: Entrapment: An important mechanism to explain the shortwave 3D
%radiative effect of clouds. \textit{J. Atmos.\ Sci.,} \textbf{76,}
%2123--2141.
%
\harvarditem{Shonk and Hogan}{2008}{Shonk+2008}Shonk, J. K. P., and
R. J. Hogan, 2008: Tripleclouds: an efficient method for representing
horizontal cloud inhomogeneity in 1D radiation schemes by using three
regions at each height. \textit{J. Climate,} \textbf{21,} 2352--2370.
%
\end{thebibliography}
\end{document}
